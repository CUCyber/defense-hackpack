\subsection{Ansible}\index{provisioner!ansible}
Ansible is a lightweight agent-less provisioner that uses Python 2.X and OpenSSH as a backend.
It is configured using Playbooks which are files written in a dialect of YAML.

\subsubsection{Setting it up}

\paragraph{Managed Windows Machines}
Ansible can also manage Windows machines running PowerShell 3.0 or later.
For windows machines, you will also need to create an encrypted host\_vars or group\_vars file on the control machine that contains the following information:

\acmlisting[caption=Ansible Windows, label=Ansible Windows]{./tools/provisioners/ansible/scripts/ansible-vault.sh}

It is also required to run the ConfigureRemotingForAnsible.ps1 from the Ansible source code script on the Windows machines that will be managed.
\paragraph{Managed Linux Machines}
Ansible is agent less, this means that only one machine must have Ansible installed on it.
For Linux or BSD managed machines, all you have to have is OpenSSH and Python 2.X.
For some Linux distributions where Python 3.X is the default, you may need to set ansible\_python\_interpetor $=$ /usr/bin/python2.

\paragraph{Control Machine}
The control machine must have a few more pieces of software.
On most distributions, it can be installed from package repositories.
It can also be installed from source:

\acmlisting[caption=Ansible Install, label=Ansible Install]{./tools/provisioners/ansible/scripts/ansible-install.sh}


\subsubsection{Inventory Management}
The collections of machines that are managed via Ansible are called the inventory.  Here is an example inventory file:

\acmlisting[caption=Ansible Hosts, label=Ansible Hosts]{./tools/provisioners/ansible/scripts/ansible-hosts.example}


In this example we demonstrate, 
\begin{itemize}
	\item A set of hosts specified by a range of ip addresses
	\item A set of hosts specified by domain name
	\item A host using a different user and default port.
	\item Targeting the localhost
	\item Four groups of hosts called web, webservers, databases, and local
	\item A group of groups called web that contains all the hosts in webservers and databases.
\end{itemize}
NOTE, group variables and host variables can also be specified in the hosts file as shown with the ansible\_connection example, but this format is discouraged because it does not follow a separation of concerns.

\subsubsection{Sample Playbooks}

Repositories containing Ansible playbooks  are generally arranged out as follows:

\begin{itemize}
\item site.yml --- the primary playbook.
\item hosts --- the primary host inventory.
\item group\_vars/ --- directory containing encrypted group variables.
\item host\_vars/ --- directory containing encrypted host variables.
\item roles/ --- directory containing roles that will be applied.
\end{itemize}

Here is an example playbook:
\acmlisting[caption=Ansible Playbook, label=Ansible Playbook, ]{tools/provisioners/ansible/scripts/ansible-playbook.yml}

These playbooks can also be split into separate sections in what are called roles.
Here is how a sample role, in this case a webserver are stored:
\begin{itemize}
\item roles/webserver --- directory where the webserver role is stored.
\item roles/webserver/files --- files that would be referenced via copy commands in the role.
\item roles/webserver/tempates --- templates that would be referenced via template commands in the role.
\item roles/webserver/tasks --- where tasks for the webserver role are stored.
\item roles/webserver/handlers --- where handlers that kickoff post processing tasks for the webserver role are stored.
\item roles/webserver/vars --- where role specific variables for the webserver role are stored.
\item roles/webserver/defaults --- where role specific default values variables for the webserver role are stored.
\item roles/webserver/meta --- where role specific meta data for the webserver role are stored such as dependencies could be listed.
\end{itemize}

