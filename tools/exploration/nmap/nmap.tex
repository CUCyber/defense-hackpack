\subsection{Nmap}\index{exploration!nmap}
Nmap is a Network exploration tool and security and port scanner.
It is useful for preforming host enumeration in IPv4 networks and auditing ports and services on specific hosts on IPv4 and IPv6 networks.

\subsubsection{Common Options}

\acmlisting[caption=nmap examples, label=nmap examples]{./tools/exploration/nmap/scripts/examples.sh}

\paragraph{Host Discovery Options}
For host discovery, the most important flag is -sn flag is very useful.
It sends an ICMP ECHO to each target host.
In IPv4 networks, this is a fast and easy way to enumerate hosts for a deeper scan.

In IPv6 networks, the address space is probably too large to do this effectively.
On solution in this case to examine the network switch MAC table or to use tcpdump or wireshark to sniff for packets.

To conduct discovery using different types of packets use the -P\{n,S,A,U,Y\} options which use no pings, SYN, ACK, UDP, and SCTP packets respectively.

\paragraph{Port Scanning Options}
By default, nmap scans the 1000 most commonly used ports.
To use nmap to scan for specific ports, use the -p flag to specify which ports to scan.
It accepts hyphen separated ranges and comma separated lists.  
To scan all ports, use the --allports long option.
To use a different type of packets use the -s\{S,T,A,W,U,Y\} options which test with SYN, TCP connect, Ack, UDP, and SCTP INIT  packets respectively.

\paragraph{Service Scanning Options}
There are several commons flags to use here:
\begin{itemize}
	\item -O will run OS detection against the target.
	\item -sV will run service version detection against the target.
	\item -sC will run common default scripts against the target to detect various things.
	\item --script="<script\_name>" long option a specified script or group of scripts against the targets.
	\item -A will run Enable OS detection, version detection, script scanning, and traceroute.
\end{itemize}

Scripts that are available can often be found the /usr/share/nmap directory.  Refer to these for examples on how to write scripts.

\paragraph{Timing and Optimization}
Nmap has a series of timing and optimizations that can be run.
The most useful is -T[1-5] which specifies how quickly packets are to be sent.
1 is the slowest, 5 is the fastest.
You can also specify max retries via the --max-retries long option.
You can also specify max timeout via the --host-timeout long option.

\paragraph{Evasive Options}
If you are running nmap offensively, there are several flags that control how evasive nmap behaves.
These allow for spoofing of IP address(-S) and mac address (--spoof-mac).
And setting various options for sending custom packets.

\paragraph{Output Options}
There are various output options the most important are:
\begin{itemize}
	\item -oN <file\_name> output normal output to a file
	\item -oG <file\_name> output grep able output to a file
	\item -oX <file\_name> output XML output to a file
\end{itemize}

