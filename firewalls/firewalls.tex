\index{firewall}
Firewalls are essentially sets of rules that allow network traffic in and out of a machine.
In general, firewalls should be configured to allow the minimum required access.
For Windows, the firewall is called Windows Firewall.
For Linux, iptables and firewalld are the most common firewalls.
For BSD, pf --- the basis of the pfsense enterprise firewall --- is the default.
Instructions for the Cisco ASA firewalls have also been included.

In general, there are 3 major elements of firewall security:

\begin{itemize}
	\item Use a default reject policy to avoid admitting unwanted traffic.
	\item Open only the required ports to make the services to work.
	\item Log any unusual traffic that hits the firewall.
\end{itemize}

\section{Iptables}\index{firewall!iptables}

Iptables stores the majority of its configuration in a series of files:
\begin{itemize}
	\item /etc/sysconfig/iptables --- where the iptables configuration stored in RedHat
		based distributions
	\item /etc/iptables --- where the iptables configuration is stored in Debian
		based distributions
	\item /etc/services --- an optional file that maps service names to port
		numbers
\end{itemize}

Iptables uses the following binaries:
\begin{itemize}
	\item iptables --- view and modify the firewall
	\item iptables-save --- prints the running configuration to stdout; Used to
		save the running configuration to a file.
	\item iptables-restore --- reads a file and sets the firewall configuration
\end{itemize}

While iptables does have a save file format, it is often configured via bash and the iptables command then saved and restored from the save file format after initial changes.
This is to avoid differences in the save file format between different version.
Iptables will stop processing a packet when it matches the first rule.
The only exception to this is the LOG target.
When the LOG target is matched, matching will continue; but the traffic will be
logged in the kernel log.

\acmlisting[caption=Iptables Script, label=Iptables Script]{./firewalls/configs/iptables.example}

\section{Firewalld}\index{firewall!firewalld}
firewalld references the following directories of files:

\begin{itemize}
	\item /usr/lib/firewalld --- where package default rules reside.
	\item /etc/firewalld --- where user overrides rules reside.
\end{itemize}

firewalld uses only the firewall-cmd binary.

firewalld is the new Linux firewall.
It provides a usability layer on top of iptables.
It focuses on zones and services.
Zones are affiliated with source address or interfaces.
Services define the ports that will be used with an application
It can be configured via scripts:

\acmlisting[caption=firewalld Script, label=firewalld Script]{./firewalls/configs/firewalld.example}

\section{pf}\index{firewall!pf}

pf references the following files:
\begin{itemize}
	\item /etc/rc.conf --- Like all bsd services, the pf service must be enabled here
	\item pf.conf  --- pf stores its configuration in the file.
\end{itemize}

Pf uses the following binaries
\begin{itemize}
	\item pfctl -f /etc/pf.conf --- load the firewall configuration
	\item pfctl -sa --- see the current configuration status
	\item kldload pf --- load the pf kernel module
\end{itemize}

All of the configuration for pf is stored in /etc/pf.conf.
There is no way to modify the running configuration except to overwrite the running
configuration with the saved configuration.
Unlike other firewalls, the last rule to match will be the rule that is applied.
This behavior can be overridden by using the `quick' keyword.

\acmlisting[caption=pf.conf, label=pf.conf]{./firewalls/configs/pf.conf.example}

