\section{logindefs}\index{logindefs}\index{linux}

Login.defs is a configuration file that controls login functionality on GNU/Linux machines using encrypted password files.
It is one of 3 tools that can control this process (login.defs, pam, logind)

In general, there are a few best practices for using login.defs:

The following settings should be configured in \lstinline|/etc/login.defs|:
\begin{itemize}
	\item \lstinline|CONSOLE| should be set to \lstinline|/etc/securetty|
	\item set \lstinline|PASS_MAX_DAYS| to 30
	\item set \lstinline|PASS_MIN_DAYS| to 7
	\item set \lstinline|PASS_WARN_DAYS| to 8
	\item set \lstinline|PASS_MIN_LEN| to 8
	\item set \lstinline|MAIL_DIR| to \lstinline|/var/spool/mail|
	\item set \lstinline|UMASK| to 077
\end{itemize}

If you set CONSOLE to, you should also ensure the following is in \lstinline|/etc/securetty|:
\begin{itemize}
	\item console
	\item tty1
	\item tty2
	\item tty3
	\item tty4
	\item tty5
	\item tty6
\end{itemize}
