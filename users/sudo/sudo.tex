\section{sudo}\index{sudo}\index{linux}

Sudo is short for Super User DO.
It allows for non-root users to request elevated privileges on linux systems.
Sudo has a variety of options that can be configured.
Here are some basic suggestions for managing systems that use sudo

\begin{itemize}
	\item Limit users permissions where possible, users should not have \lstinline|ALL = (ALL) ALL|
	\item Avoid using groups with root permissions
\end{itemize}

The sudoers (config file for sudo) file should be edited as root using the visudo command which verifies the syntax before making changes.
There are also other configuration files that can be found in \lstinline|/etc/sudoers.d|
These can be edited using \lstinline|visudo -f <filename>|

The following script configures sudoers with an appropriate:

\lstinputlisting{users/scripts/sudoers.sh}
